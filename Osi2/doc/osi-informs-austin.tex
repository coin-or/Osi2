\documentclass{beamer}

\title{Next Steps for the  \\ COIN-OR \\ Open Solver Interface}
\author{Lou Hafer\inst{1} \and Matthew Saltzman\inst{2}}
\institute{
  \inst{1}
  Department of Computer Science \\
  Simon Fraser University
  \and
  \inst{2}
  Department of Mathematical Sciences \\
  Clemson University.
}
\date{INFORMS Annual Meeting \\ Austin, Texas \\ November 10, 2010}

\begin{document}
\begin{frame}
  \titlepage
\end{frame}

\begin{frame}
  \frametitle{Outline}

  \begin{itemize}
  \item What's wrong with the current API?
  \item Dynamic Plugin Architecture
  \item Factories and Run-Time Instantiation
  \item Callbacks: Function-Pointer and Object-Based Approaches
  \item Extending the Basic Interface: OsiSimplex, etc.
  \end{itemize}
\end{frame}

\begin{frame}
  \frametitle{What's Wrong with the Current OSI?}

  The current OSI architecture is the original one from 2000.  It
  has proved to be useful in many contexts, but it has several
  properties that make it difficult to use and maintain.
  \begin{itemize}
  \item Back-end interfaces are created by inheritance from a base
    class (front end).
    \begin{itemize}
    \item Front-end and back-end interfaces can get out of sync.
    \item Too many tasks are implemented in the back end.
    \item Extending interfaces is difficult.
    \item Separation between instances and algorithms is problematic.
    \end{itemize}
  \item Solver interfaces are required at compile time.
    \begin{itemize}
    \item Control of which interfaces are needed is handled in
      source code, or by preprocessor directives (\texttt{\#ifdef}).
    \end{itemize}
  \item Solver libraries are required at link time.
  \end{itemize}
\end{frame}

\begin{frame}
  \frametitle{Plugin Architecture for Dynamic Solver Engine Linking}
  \begin{itemize}
  \item Decide which back-end interface and solver engine at runtime.
  \item Load just functions needed.
  \item No need to include engine library references with binary
    distributions.
  \item Windows DLLs and *nix static, shared, and dynamic libraries
    can be handled.
  \item C++ and C interfaces are possible.  With C++, all class
    information is exposed across the interface.  With C, objects
    can be wrapped to pass across.
  \end{itemize}
\end{frame}

\begin{frame}
  \frametitle{Callbacks and Event Handlers}

  Opprtunity for the user to collect information about algorithm
  progress or take action at certain points in the algorithm or on
  the occurence of certain external events.
  \begin{itemize}
  \item C solvers allow the user to register functions with the
    solver to be called at those points.
  \item C++ solvers call virtual functions at those points.  Users
    can derive replacement functions to achieve the desired result.
  \item A portable callback framework is a challenge.  Full callback
    flexibility probably requires exposing the solver engine's
    callback interface.
  \item Portability for callbacks using the intersection of common
    callback interfaces can be done.
  \item Hiding the C mechanism from C++ programs is possible.
    Hiding the C++ mechanism from C programs is possible if the set
    of actions is fixed (e.g., just abort).
  \end{itemize}
\end{frame}
\end{document}
